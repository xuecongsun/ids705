
% Default to the notebook output style

    


% Inherit from the specified cell style.




    
\documentclass[11pt]{article}

    
    
    \usepackage[T1]{fontenc}
    % Nicer default font (+ math font) than Computer Modern for most use cases
    \usepackage{mathpazo}

    % Basic figure setup, for now with no caption control since it's done
    % automatically by Pandoc (which extracts ![](path) syntax from Markdown).
    \usepackage{graphicx}
    % We will generate all images so they have a width \maxwidth. This means
    % that they will get their normal width if they fit onto the page, but
    % are scaled down if they would overflow the margins.
    \makeatletter
    \def\maxwidth{\ifdim\Gin@nat@width>\linewidth\linewidth
    \else\Gin@nat@width\fi}
    \makeatother
    \let\Oldincludegraphics\includegraphics
    % Set max figure width to be 80% of text width, for now hardcoded.
    \renewcommand{\includegraphics}[1]{\Oldincludegraphics[width=.8\maxwidth]{#1}}
    % Ensure that by default, figures have no caption (until we provide a
    % proper Figure object with a Caption API and a way to capture that
    % in the conversion process - todo).
    \usepackage{caption}
    \DeclareCaptionLabelFormat{nolabel}{}
    \captionsetup{labelformat=nolabel}

    \usepackage{adjustbox} % Used to constrain images to a maximum size 
    \usepackage{xcolor} % Allow colors to be defined
    \usepackage{enumerate} % Needed for markdown enumerations to work
    \usepackage{geometry} % Used to adjust the document margins
    \usepackage{amsmath} % Equations
    \usepackage{amssymb} % Equations
    \usepackage{textcomp} % defines textquotesingle
    % Hack from http://tex.stackexchange.com/a/47451/13684:
    \AtBeginDocument{%
        \def\PYZsq{\textquotesingle}% Upright quotes in Pygmentized code
    }
    \usepackage{upquote} % Upright quotes for verbatim code
    \usepackage{eurosym} % defines \euro
    \usepackage[mathletters]{ucs} % Extended unicode (utf-8) support
    \usepackage[utf8x]{inputenc} % Allow utf-8 characters in the tex document
    \usepackage{fancyvrb} % verbatim replacement that allows latex
    \usepackage{grffile} % extends the file name processing of package graphics 
                         % to support a larger range 
    % The hyperref package gives us a pdf with properly built
    % internal navigation ('pdf bookmarks' for the table of contents,
    % internal cross-reference links, web links for URLs, etc.)
    \usepackage{hyperref}
    \usepackage{longtable} % longtable support required by pandoc >1.10
    \usepackage{booktabs}  % table support for pandoc > 1.12.2
    \usepackage[inline]{enumitem} % IRkernel/repr support (it uses the enumerate* environment)
    \usepackage[normalem]{ulem} % ulem is needed to support strikethroughs (\sout)
                                % normalem makes italics be italics, not underlines
    

    
    
    % Colors for the hyperref package
    \definecolor{urlcolor}{rgb}{0,.145,.698}
    \definecolor{linkcolor}{rgb}{.71,0.21,0.01}
    \definecolor{citecolor}{rgb}{.12,.54,.11}

    % ANSI colors
    \definecolor{ansi-black}{HTML}{3E424D}
    \definecolor{ansi-black-intense}{HTML}{282C36}
    \definecolor{ansi-red}{HTML}{E75C58}
    \definecolor{ansi-red-intense}{HTML}{B22B31}
    \definecolor{ansi-green}{HTML}{00A250}
    \definecolor{ansi-green-intense}{HTML}{007427}
    \definecolor{ansi-yellow}{HTML}{DDB62B}
    \definecolor{ansi-yellow-intense}{HTML}{B27D12}
    \definecolor{ansi-blue}{HTML}{208FFB}
    \definecolor{ansi-blue-intense}{HTML}{0065CA}
    \definecolor{ansi-magenta}{HTML}{D160C4}
    \definecolor{ansi-magenta-intense}{HTML}{A03196}
    \definecolor{ansi-cyan}{HTML}{60C6C8}
    \definecolor{ansi-cyan-intense}{HTML}{258F8F}
    \definecolor{ansi-white}{HTML}{C5C1B4}
    \definecolor{ansi-white-intense}{HTML}{A1A6B2}

    % commands and environments needed by pandoc snippets
    % extracted from the output of `pandoc -s`
    \providecommand{\tightlist}{%
      \setlength{\itemsep}{0pt}\setlength{\parskip}{0pt}}
    \DefineVerbatimEnvironment{Highlighting}{Verbatim}{commandchars=\\\{\}}
    % Add ',fontsize=\small' for more characters per line
    \newenvironment{Shaded}{}{}
    \newcommand{\KeywordTok}[1]{\textcolor[rgb]{0.00,0.44,0.13}{\textbf{{#1}}}}
    \newcommand{\DataTypeTok}[1]{\textcolor[rgb]{0.56,0.13,0.00}{{#1}}}
    \newcommand{\DecValTok}[1]{\textcolor[rgb]{0.25,0.63,0.44}{{#1}}}
    \newcommand{\BaseNTok}[1]{\textcolor[rgb]{0.25,0.63,0.44}{{#1}}}
    \newcommand{\FloatTok}[1]{\textcolor[rgb]{0.25,0.63,0.44}{{#1}}}
    \newcommand{\CharTok}[1]{\textcolor[rgb]{0.25,0.44,0.63}{{#1}}}
    \newcommand{\StringTok}[1]{\textcolor[rgb]{0.25,0.44,0.63}{{#1}}}
    \newcommand{\CommentTok}[1]{\textcolor[rgb]{0.38,0.63,0.69}{\textit{{#1}}}}
    \newcommand{\OtherTok}[1]{\textcolor[rgb]{0.00,0.44,0.13}{{#1}}}
    \newcommand{\AlertTok}[1]{\textcolor[rgb]{1.00,0.00,0.00}{\textbf{{#1}}}}
    \newcommand{\FunctionTok}[1]{\textcolor[rgb]{0.02,0.16,0.49}{{#1}}}
    \newcommand{\RegionMarkerTok}[1]{{#1}}
    \newcommand{\ErrorTok}[1]{\textcolor[rgb]{1.00,0.00,0.00}{\textbf{{#1}}}}
    \newcommand{\NormalTok}[1]{{#1}}
    
    % Additional commands for more recent versions of Pandoc
    \newcommand{\ConstantTok}[1]{\textcolor[rgb]{0.53,0.00,0.00}{{#1}}}
    \newcommand{\SpecialCharTok}[1]{\textcolor[rgb]{0.25,0.44,0.63}{{#1}}}
    \newcommand{\VerbatimStringTok}[1]{\textcolor[rgb]{0.25,0.44,0.63}{{#1}}}
    \newcommand{\SpecialStringTok}[1]{\textcolor[rgb]{0.73,0.40,0.53}{{#1}}}
    \newcommand{\ImportTok}[1]{{#1}}
    \newcommand{\DocumentationTok}[1]{\textcolor[rgb]{0.73,0.13,0.13}{\textit{{#1}}}}
    \newcommand{\AnnotationTok}[1]{\textcolor[rgb]{0.38,0.63,0.69}{\textbf{\textit{{#1}}}}}
    \newcommand{\CommentVarTok}[1]{\textcolor[rgb]{0.38,0.63,0.69}{\textbf{\textit{{#1}}}}}
    \newcommand{\VariableTok}[1]{\textcolor[rgb]{0.10,0.09,0.49}{{#1}}}
    \newcommand{\ControlFlowTok}[1]{\textcolor[rgb]{0.00,0.44,0.13}{\textbf{{#1}}}}
    \newcommand{\OperatorTok}[1]{\textcolor[rgb]{0.40,0.40,0.40}{{#1}}}
    \newcommand{\BuiltInTok}[1]{{#1}}
    \newcommand{\ExtensionTok}[1]{{#1}}
    \newcommand{\PreprocessorTok}[1]{\textcolor[rgb]{0.74,0.48,0.00}{{#1}}}
    \newcommand{\AttributeTok}[1]{\textcolor[rgb]{0.49,0.56,0.16}{{#1}}}
    \newcommand{\InformationTok}[1]{\textcolor[rgb]{0.38,0.63,0.69}{\textbf{\textit{{#1}}}}}
    \newcommand{\WarningTok}[1]{\textcolor[rgb]{0.38,0.63,0.69}{\textbf{\textit{{#1}}}}}
    
    
    % Define a nice break command that doesn't care if a line doesn't already
    % exist.
    \def\br{\hspace*{\fill} \\* }
    % Math Jax compatability definitions
    \def\gt{>}
    \def\lt{<}
    % Document parameters
    \title{Assignment 1}
    
    
    

    % Pygments definitions
    
\makeatletter
\def\PY@reset{\let\PY@it=\relax \let\PY@bf=\relax%
    \let\PY@ul=\relax \let\PY@tc=\relax%
    \let\PY@bc=\relax \let\PY@ff=\relax}
\def\PY@tok#1{\csname PY@tok@#1\endcsname}
\def\PY@toks#1+{\ifx\relax#1\empty\else%
    \PY@tok{#1}\expandafter\PY@toks\fi}
\def\PY@do#1{\PY@bc{\PY@tc{\PY@ul{%
    \PY@it{\PY@bf{\PY@ff{#1}}}}}}}
\def\PY#1#2{\PY@reset\PY@toks#1+\relax+\PY@do{#2}}

\expandafter\def\csname PY@tok@w\endcsname{\def\PY@tc##1{\textcolor[rgb]{0.73,0.73,0.73}{##1}}}
\expandafter\def\csname PY@tok@c\endcsname{\let\PY@it=\textit\def\PY@tc##1{\textcolor[rgb]{0.25,0.50,0.50}{##1}}}
\expandafter\def\csname PY@tok@cp\endcsname{\def\PY@tc##1{\textcolor[rgb]{0.74,0.48,0.00}{##1}}}
\expandafter\def\csname PY@tok@k\endcsname{\let\PY@bf=\textbf\def\PY@tc##1{\textcolor[rgb]{0.00,0.50,0.00}{##1}}}
\expandafter\def\csname PY@tok@kp\endcsname{\def\PY@tc##1{\textcolor[rgb]{0.00,0.50,0.00}{##1}}}
\expandafter\def\csname PY@tok@kt\endcsname{\def\PY@tc##1{\textcolor[rgb]{0.69,0.00,0.25}{##1}}}
\expandafter\def\csname PY@tok@o\endcsname{\def\PY@tc##1{\textcolor[rgb]{0.40,0.40,0.40}{##1}}}
\expandafter\def\csname PY@tok@ow\endcsname{\let\PY@bf=\textbf\def\PY@tc##1{\textcolor[rgb]{0.67,0.13,1.00}{##1}}}
\expandafter\def\csname PY@tok@nb\endcsname{\def\PY@tc##1{\textcolor[rgb]{0.00,0.50,0.00}{##1}}}
\expandafter\def\csname PY@tok@nf\endcsname{\def\PY@tc##1{\textcolor[rgb]{0.00,0.00,1.00}{##1}}}
\expandafter\def\csname PY@tok@nc\endcsname{\let\PY@bf=\textbf\def\PY@tc##1{\textcolor[rgb]{0.00,0.00,1.00}{##1}}}
\expandafter\def\csname PY@tok@nn\endcsname{\let\PY@bf=\textbf\def\PY@tc##1{\textcolor[rgb]{0.00,0.00,1.00}{##1}}}
\expandafter\def\csname PY@tok@ne\endcsname{\let\PY@bf=\textbf\def\PY@tc##1{\textcolor[rgb]{0.82,0.25,0.23}{##1}}}
\expandafter\def\csname PY@tok@nv\endcsname{\def\PY@tc##1{\textcolor[rgb]{0.10,0.09,0.49}{##1}}}
\expandafter\def\csname PY@tok@no\endcsname{\def\PY@tc##1{\textcolor[rgb]{0.53,0.00,0.00}{##1}}}
\expandafter\def\csname PY@tok@nl\endcsname{\def\PY@tc##1{\textcolor[rgb]{0.63,0.63,0.00}{##1}}}
\expandafter\def\csname PY@tok@ni\endcsname{\let\PY@bf=\textbf\def\PY@tc##1{\textcolor[rgb]{0.60,0.60,0.60}{##1}}}
\expandafter\def\csname PY@tok@na\endcsname{\def\PY@tc##1{\textcolor[rgb]{0.49,0.56,0.16}{##1}}}
\expandafter\def\csname PY@tok@nt\endcsname{\let\PY@bf=\textbf\def\PY@tc##1{\textcolor[rgb]{0.00,0.50,0.00}{##1}}}
\expandafter\def\csname PY@tok@nd\endcsname{\def\PY@tc##1{\textcolor[rgb]{0.67,0.13,1.00}{##1}}}
\expandafter\def\csname PY@tok@s\endcsname{\def\PY@tc##1{\textcolor[rgb]{0.73,0.13,0.13}{##1}}}
\expandafter\def\csname PY@tok@sd\endcsname{\let\PY@it=\textit\def\PY@tc##1{\textcolor[rgb]{0.73,0.13,0.13}{##1}}}
\expandafter\def\csname PY@tok@si\endcsname{\let\PY@bf=\textbf\def\PY@tc##1{\textcolor[rgb]{0.73,0.40,0.53}{##1}}}
\expandafter\def\csname PY@tok@se\endcsname{\let\PY@bf=\textbf\def\PY@tc##1{\textcolor[rgb]{0.73,0.40,0.13}{##1}}}
\expandafter\def\csname PY@tok@sr\endcsname{\def\PY@tc##1{\textcolor[rgb]{0.73,0.40,0.53}{##1}}}
\expandafter\def\csname PY@tok@ss\endcsname{\def\PY@tc##1{\textcolor[rgb]{0.10,0.09,0.49}{##1}}}
\expandafter\def\csname PY@tok@sx\endcsname{\def\PY@tc##1{\textcolor[rgb]{0.00,0.50,0.00}{##1}}}
\expandafter\def\csname PY@tok@m\endcsname{\def\PY@tc##1{\textcolor[rgb]{0.40,0.40,0.40}{##1}}}
\expandafter\def\csname PY@tok@gh\endcsname{\let\PY@bf=\textbf\def\PY@tc##1{\textcolor[rgb]{0.00,0.00,0.50}{##1}}}
\expandafter\def\csname PY@tok@gu\endcsname{\let\PY@bf=\textbf\def\PY@tc##1{\textcolor[rgb]{0.50,0.00,0.50}{##1}}}
\expandafter\def\csname PY@tok@gd\endcsname{\def\PY@tc##1{\textcolor[rgb]{0.63,0.00,0.00}{##1}}}
\expandafter\def\csname PY@tok@gi\endcsname{\def\PY@tc##1{\textcolor[rgb]{0.00,0.63,0.00}{##1}}}
\expandafter\def\csname PY@tok@gr\endcsname{\def\PY@tc##1{\textcolor[rgb]{1.00,0.00,0.00}{##1}}}
\expandafter\def\csname PY@tok@ge\endcsname{\let\PY@it=\textit}
\expandafter\def\csname PY@tok@gs\endcsname{\let\PY@bf=\textbf}
\expandafter\def\csname PY@tok@gp\endcsname{\let\PY@bf=\textbf\def\PY@tc##1{\textcolor[rgb]{0.00,0.00,0.50}{##1}}}
\expandafter\def\csname PY@tok@go\endcsname{\def\PY@tc##1{\textcolor[rgb]{0.53,0.53,0.53}{##1}}}
\expandafter\def\csname PY@tok@gt\endcsname{\def\PY@tc##1{\textcolor[rgb]{0.00,0.27,0.87}{##1}}}
\expandafter\def\csname PY@tok@err\endcsname{\def\PY@bc##1{\setlength{\fboxsep}{0pt}\fcolorbox[rgb]{1.00,0.00,0.00}{1,1,1}{\strut ##1}}}
\expandafter\def\csname PY@tok@kc\endcsname{\let\PY@bf=\textbf\def\PY@tc##1{\textcolor[rgb]{0.00,0.50,0.00}{##1}}}
\expandafter\def\csname PY@tok@kd\endcsname{\let\PY@bf=\textbf\def\PY@tc##1{\textcolor[rgb]{0.00,0.50,0.00}{##1}}}
\expandafter\def\csname PY@tok@kn\endcsname{\let\PY@bf=\textbf\def\PY@tc##1{\textcolor[rgb]{0.00,0.50,0.00}{##1}}}
\expandafter\def\csname PY@tok@kr\endcsname{\let\PY@bf=\textbf\def\PY@tc##1{\textcolor[rgb]{0.00,0.50,0.00}{##1}}}
\expandafter\def\csname PY@tok@bp\endcsname{\def\PY@tc##1{\textcolor[rgb]{0.00,0.50,0.00}{##1}}}
\expandafter\def\csname PY@tok@fm\endcsname{\def\PY@tc##1{\textcolor[rgb]{0.00,0.00,1.00}{##1}}}
\expandafter\def\csname PY@tok@vc\endcsname{\def\PY@tc##1{\textcolor[rgb]{0.10,0.09,0.49}{##1}}}
\expandafter\def\csname PY@tok@vg\endcsname{\def\PY@tc##1{\textcolor[rgb]{0.10,0.09,0.49}{##1}}}
\expandafter\def\csname PY@tok@vi\endcsname{\def\PY@tc##1{\textcolor[rgb]{0.10,0.09,0.49}{##1}}}
\expandafter\def\csname PY@tok@vm\endcsname{\def\PY@tc##1{\textcolor[rgb]{0.10,0.09,0.49}{##1}}}
\expandafter\def\csname PY@tok@sa\endcsname{\def\PY@tc##1{\textcolor[rgb]{0.73,0.13,0.13}{##1}}}
\expandafter\def\csname PY@tok@sb\endcsname{\def\PY@tc##1{\textcolor[rgb]{0.73,0.13,0.13}{##1}}}
\expandafter\def\csname PY@tok@sc\endcsname{\def\PY@tc##1{\textcolor[rgb]{0.73,0.13,0.13}{##1}}}
\expandafter\def\csname PY@tok@dl\endcsname{\def\PY@tc##1{\textcolor[rgb]{0.73,0.13,0.13}{##1}}}
\expandafter\def\csname PY@tok@s2\endcsname{\def\PY@tc##1{\textcolor[rgb]{0.73,0.13,0.13}{##1}}}
\expandafter\def\csname PY@tok@sh\endcsname{\def\PY@tc##1{\textcolor[rgb]{0.73,0.13,0.13}{##1}}}
\expandafter\def\csname PY@tok@s1\endcsname{\def\PY@tc##1{\textcolor[rgb]{0.73,0.13,0.13}{##1}}}
\expandafter\def\csname PY@tok@mb\endcsname{\def\PY@tc##1{\textcolor[rgb]{0.40,0.40,0.40}{##1}}}
\expandafter\def\csname PY@tok@mf\endcsname{\def\PY@tc##1{\textcolor[rgb]{0.40,0.40,0.40}{##1}}}
\expandafter\def\csname PY@tok@mh\endcsname{\def\PY@tc##1{\textcolor[rgb]{0.40,0.40,0.40}{##1}}}
\expandafter\def\csname PY@tok@mi\endcsname{\def\PY@tc##1{\textcolor[rgb]{0.40,0.40,0.40}{##1}}}
\expandafter\def\csname PY@tok@il\endcsname{\def\PY@tc##1{\textcolor[rgb]{0.40,0.40,0.40}{##1}}}
\expandafter\def\csname PY@tok@mo\endcsname{\def\PY@tc##1{\textcolor[rgb]{0.40,0.40,0.40}{##1}}}
\expandafter\def\csname PY@tok@ch\endcsname{\let\PY@it=\textit\def\PY@tc##1{\textcolor[rgb]{0.25,0.50,0.50}{##1}}}
\expandafter\def\csname PY@tok@cm\endcsname{\let\PY@it=\textit\def\PY@tc##1{\textcolor[rgb]{0.25,0.50,0.50}{##1}}}
\expandafter\def\csname PY@tok@cpf\endcsname{\let\PY@it=\textit\def\PY@tc##1{\textcolor[rgb]{0.25,0.50,0.50}{##1}}}
\expandafter\def\csname PY@tok@c1\endcsname{\let\PY@it=\textit\def\PY@tc##1{\textcolor[rgb]{0.25,0.50,0.50}{##1}}}
\expandafter\def\csname PY@tok@cs\endcsname{\let\PY@it=\textit\def\PY@tc##1{\textcolor[rgb]{0.25,0.50,0.50}{##1}}}

\def\PYZbs{\char`\\}
\def\PYZus{\char`\_}
\def\PYZob{\char`\{}
\def\PYZcb{\char`\}}
\def\PYZca{\char`\^}
\def\PYZam{\char`\&}
\def\PYZlt{\char`\<}
\def\PYZgt{\char`\>}
\def\PYZsh{\char`\#}
\def\PYZpc{\char`\%}
\def\PYZdl{\char`\$}
\def\PYZhy{\char`\-}
\def\PYZsq{\char`\'}
\def\PYZdq{\char`\"}
\def\PYZti{\char`\~}
% for compatibility with earlier versions
\def\PYZat{@}
\def\PYZlb{[}
\def\PYZrb{]}
\makeatother


    % Exact colors from NB
    \definecolor{incolor}{rgb}{0.0, 0.0, 0.5}
    \definecolor{outcolor}{rgb}{0.545, 0.0, 0.0}



    
    % Prevent overflowing lines due to hard-to-break entities
    \sloppy 
    % Setup hyperref package
    \hypersetup{
      breaklinks=true,  % so long urls are correctly broken across lines
      colorlinks=true,
      urlcolor=urlcolor,
      linkcolor=linkcolor,
      citecolor=citecolor,
      }
    % Slightly bigger margins than the latex defaults
    
    \geometry{verbose,tmargin=1in,bmargin=1in,lmargin=1in,rmargin=1in}
    
    

    \begin{document}
    
    
    \maketitle
    
    

    
    \section{Assignment 1 - Probability, Linear Algebra, Programming, and
Git}\label{assignment-1---probability-linear-algebra-programming-and-git}

\subsection{\texorpdfstring{\emph{Emma Sun
Xuecong}}{Emma Sun Xuecong}}\label{emma-sun-xuecong}

Netid: \emph{xs58}

    \section{Probability and Statistics
Theory}\label{probability-and-statistics-theory}

    \subsection{1}\label{section}

Let
\(f(x) = \begin{cases}  0 & x < 0 \\  \alpha x^2 & 0 \leq x \leq 2 \\  0 & 2 < x  \end{cases}\)

For what value of \(\alpha\) is \(f(x)\) a valid probability density
function?

\emph{Note: for all assignments, write out all equations and math for
all assignments using markdown and
\href{https://tobi.oetiker.ch/lshort/lshort.pdf}{LaTeX} and show all
work}

    \textbf{ANSWER}

    For \(f(x)\) to be a valid probability density function, it has to
satisfy the condition that: \[\int_{-\infty}^{\infty} f(x) dx\]

Therefore, in this case,

\[
\begin{align}
\int_{0}^{2} \alpha x^2 dx=1 \\
(\frac{1}{3}\alpha x^3)_{0}^{2}=1 \\
\frac{8}{3}\alpha=1\\
\alpha=\frac{3}{8}
\end{align}
\]

    \subsection{2}\label{section}

What is the cumulative distribution function (CDF) that corresponds to
the following probability distribution function? Please state the value
of the CDF for all possible values of \(x\).

\(f(x) = \begin{cases}  \frac{1}{3} & 0 < x < 3 \\  0 & \text{otherwise}  \end{cases}\)

    \textbf{ANSWER}

    \(P(x)=\int_{-\infty}^{x} f(x) dx  = \begin{cases}  0 & x<0 \\  \frac{x}{3} & 0 < x < 3 \\  1 & x \geq 3  \end{cases}\)

    \subsection{3}\label{section}

For the probability distribution function for the random variable \(X\),

\(f(x) = \begin{cases}  \frac{1}{3} & 0 < x < 3 \\  0 & \text{otherwise}  \end{cases}\)

what is the (a) expected value and (b) variance of \(X\). \emph{Show all
work}.

    \textbf{ANSWER}

\begin{enumerate}
\def\labelenumi{(\alph{enumi})}
\item
  Expected value:
\item
  Variance:
\end{enumerate}

    \begin{enumerate}
\def\labelenumi{(\alph{enumi})}
\tightlist
\item
  Expected Value:
\end{enumerate}

\[
\begin{align}
E(X)&= \int_{0}^{3} x f(x) dx \\
    &=\int_{0}^{3} x \frac{1}{3} dx \\ 
    &=(\frac{1}{6} x^2)_{0}^{3}\\
    &=\frac{3}{2}
\end{align}
\]

\begin{enumerate}
\def\labelenumi{(\alph{enumi})}
\setcounter{enumi}{1}
\tightlist
\item
  Variance:
\end{enumerate}

\[
\begin{align}
Var(X)&=E[X^2]-E[X]^2 \\
      &=\int_{0}^{3} x^2 f(x) dx - (\frac{3}{2})^2\\
      &=\int_{0}^{3} x^2 \frac{1}{3} dx - (\frac{3}{2})^2\\
      &=(\frac{1}{9} x^3)_{0}^{3}- (\frac{3}{2})^2\\
      &=3-\frac{9}{4}\\
      &=\frac{3}{4}
\end{align}
\]

    \subsection{4}\label{section}

Consider the following table of data that provides the values of a
discrete data vector \(\mathbf{x}\) of samples from the random variable
\(X\), where each entry in \(\mathbf{x}\) is given as \(x_i\).

\emph{Table 1. Dataset N=5 observations}

\begin{longtable}[]{@{}llllll@{}}
\toprule
& \(x_0\) & \(x_1\) & \(x_2\) & \(x_3\) & \(x_4\)\tabularnewline
\midrule
\endhead
\(\textbf{x}\) & 2 & 3 & 10 & -1 & -1\tabularnewline
\bottomrule
\end{longtable}

What is the (a) mean, (b) variance, and the of the data?

\emph{Show all work. Your answer should include the definition of mean,
median, and variance in the context of discrete data.}

    \textbf{ANSWER}

\begin{enumerate}
\def\labelenumi{(\alph{enumi})}
\item
  Mean
\item
  Variance
\end{enumerate}

    (a)Mean \[
\begin{align}
E[X] &= \frac{1}{N}\sum_{1}^{N}x_{i} \\
     &=\frac{1}{5}(2+3+10-1-1) \\
     &=\frac{13}{5}
\end{align}
\]

(b)Variance

Denote mean using \(\mu\)

\[
\begin{align}
Var[X] &= \frac{1}{N}\sum_{1}^{n}(x_{i}-\mu)^2 \\
       &= \frac{1}{5}((\frac{13}{5}-2)^2+(\frac{13}{5}-3)^2+(\frac{13}{5}-10)^2+(\frac{13}{5}-(-1))^2+(-1-\frac{13}{5})^2) \\
       &=16.24
\end{align}
\]

    \subsection{5}\label{section}

Review of counting from probability theory.

\begin{enumerate}
\def\labelenumi{(\alph{enumi})}
\item
  How many different 7-place license plates are possible if the first 3
  places only contain letters and the last 4 only contain numbers?
\item
  How many different batting orders are possible for a baseball team
  with 9 players?
\item
  How many batting orders of 5 players are possible for a team with 9
  players total?
\item
  Let's assume this class has 26 students and we want to form project
  teams. How many unique teams of 3 are possible?
\end{enumerate}

\emph{Hint: For each problem, determine if order matters, and if it
should be calculated with or without replacement.}

    \textbf{ANSWER}

\begin{enumerate}
\def\labelenumi{(\alph{enumi})}
\item
  \(26\times26\times26\times10\times10\times10\times10=175760000\)
\item
  \(9!=362880\)
\item
  \(9\times8\times7\times6\times5=15120\)
\item
  \$\{\{26\}\choose{3}\}= \frac{26!}{23!3!} = 2600 \$
\end{enumerate}

    \section{Linear Algebra}\label{linear-algebra}

    \subsection{6}\label{section}

\textbf{Matrix manipulations and multiplication}. Machine learning
involves working with many matrices, so this exercise will provide you
with the opportunity to practice those skills.

Let
\(\mathbf{A} = \begin{bmatrix} 1 & 2 & 3 \\ 2 & 4 & 5 \\ 3 & 5 & 6 \end{bmatrix}\),
\(\mathbf{b} = \begin{bmatrix} -1 \\ 3 \\ 8 \end{bmatrix}\),
\(\mathbf{c} = \begin{bmatrix} 4 \\ -3 \\ 6 \end{bmatrix}\), and
\(\mathbf{I} = \begin{bmatrix} 1 & 0 & 0 \\ 0 & 1 & 0 \\ 0 & 0 & 1 \end{bmatrix}\)

Compute the following or indicate that it cannot be computed:

\begin{enumerate}
\def\labelenumi{\arabic{enumi}.}
\tightlist
\item
  \(\mathbf{A}\mathbf{A}\)
\item
  \(\mathbf{A}\mathbf{A}^T\)
\item
  \(\mathbf{A}\mathbf{b}\)
\item
  \(\mathbf{A}\mathbf{b}^T\)
\item
  \(\mathbf{b}\mathbf{A}\)
\item
  \(\mathbf{b}^T\mathbf{A}\)
\item
  \(\mathbf{b}\mathbf{b}\)
\item
  \(\mathbf{b}^T\mathbf{b}\)
\item
  \(\mathbf{b}\mathbf{b}^T\)
\item
  \(\mathbf{b} + \mathbf{c}^T\)
\item
  \(\mathbf{b}^T\mathbf{b}^T\)
\item
  \(\mathbf{A}^{-1}\mathbf{b}\)
\item
  \(\mathbf{A}\circ\mathbf{A}\)
\item
  \(\mathbf{b}\circ\mathbf{c}\)
\end{enumerate}

\emph{Note: The element-wise (or Hadamard) product is the product of
each element in one matrix with the corresponding element in another
matrix, and is represented by the symbol "\(\circ\)".}

    \textbf{ANSWER}

    \begin{Verbatim}[commandchars=\\\{\}]
{\color{incolor}In [{\color{incolor}55}]:} \PY{k+kn}{import} \PY{n+nn}{numpy} \PY{k}{as} \PY{n+nn}{np}
         \PY{n}{A}\PY{o}{=}\PY{n}{np}\PY{o}{.}\PY{n}{array}\PY{p}{(}\PY{p}{[}\PY{p}{[}\PY{l+m+mi}{1}\PY{p}{,}\PY{l+m+mi}{2}\PY{p}{,}\PY{l+m+mi}{3}\PY{p}{]}\PY{p}{,}
                    \PY{p}{[}\PY{l+m+mi}{2}\PY{p}{,}\PY{l+m+mi}{4}\PY{p}{,}\PY{l+m+mi}{5}\PY{p}{]}\PY{p}{,}
                    \PY{p}{[}\PY{l+m+mi}{3}\PY{p}{,}\PY{l+m+mi}{5}\PY{p}{,}\PY{l+m+mi}{6}\PY{p}{]}\PY{p}{]}\PY{p}{)}
         \PY{n}{b}\PY{o}{=}\PY{n}{np}\PY{o}{.}\PY{n}{array}\PY{p}{(}\PY{p}{[}\PY{o}{\PYZhy{}}\PY{l+m+mi}{1}\PY{p}{,}\PY{l+m+mi}{3}\PY{p}{,}\PY{l+m+mi}{8}\PY{p}{]}\PY{p}{)}
         \PY{n}{b}\PY{o}{=}\PY{n}{b}\PY{o}{.}\PY{n}{reshape}\PY{p}{(}\PY{o}{\PYZhy{}}\PY{l+m+mi}{1}\PY{p}{,}\PY{l+m+mi}{1}\PY{p}{)}
         \PY{n}{c}\PY{o}{=}\PY{n}{np}\PY{o}{.}\PY{n}{array}\PY{p}{(}\PY{p}{[}\PY{l+m+mi}{4}\PY{p}{,}\PY{o}{\PYZhy{}}\PY{l+m+mi}{3}\PY{p}{,}\PY{l+m+mi}{6}\PY{p}{]}\PY{p}{)}
         \PY{n}{c}\PY{o}{=}\PY{n}{c}\PY{o}{.}\PY{n}{reshape}\PY{p}{(}\PY{o}{\PYZhy{}}\PY{l+m+mi}{1}\PY{p}{,}\PY{l+m+mi}{1}\PY{p}{)}
         \PY{n+nb}{print}\PY{p}{(}\PY{n}{A}\PY{o}{.}\PY{n}{shape}\PY{p}{)}
         \PY{n+nb}{print}\PY{p}{(}\PY{n}{b}\PY{o}{.}\PY{n}{shape}\PY{p}{)}
         \PY{n+nb}{print}\PY{p}{(}\PY{n}{c}\PY{o}{.}\PY{n}{shape}\PY{p}{)}
         
         \PY{c+c1}{\PYZsh{}1}
         \PY{n}{answer1}\PY{o}{=}\PY{n}{A}\PY{n+nd}{@A}
         \PY{n+nb}{print}\PY{p}{(}\PY{l+s+s1}{\PYZsq{}}\PY{l+s+s1}{answer 1 }\PY{l+s+se}{\PYZbs{}n}\PY{l+s+s1}{ }\PY{l+s+si}{\PYZob{}\PYZcb{}}\PY{l+s+s1}{\PYZsq{}}\PY{o}{.}\PY{n}{format}\PY{p}{(}\PY{n}{answer1}\PY{p}{)}\PY{p}{)}
         
         \PY{c+c1}{\PYZsh{}2}
         \PY{n}{answer2}\PY{o}{=}\PY{n}{A}\PY{n+nd}{@A}\PY{o}{.}\PY{n}{T}
         \PY{n+nb}{print}\PY{p}{(}\PY{l+s+s1}{\PYZsq{}}\PY{l+s+s1}{answer 2 }\PY{l+s+se}{\PYZbs{}n}\PY{l+s+s1}{ }\PY{l+s+si}{\PYZob{}\PYZcb{}}\PY{l+s+s1}{\PYZsq{}}\PY{o}{.}\PY{n}{format}\PY{p}{(}\PY{n}{answer2}\PY{p}{)}\PY{p}{)}
         
         \PY{c+c1}{\PYZsh{}3}
         \PY{n}{answer3}\PY{o}{=}\PY{n}{A}\PY{n+nd}{@b}
         \PY{n+nb}{print}\PY{p}{(}\PY{l+s+s1}{\PYZsq{}}\PY{l+s+s1}{answer 3 }\PY{l+s+se}{\PYZbs{}n}\PY{l+s+s1}{ }\PY{l+s+si}{\PYZob{}\PYZcb{}}\PY{l+s+s1}{\PYZsq{}}\PY{o}{.}\PY{n}{format}\PY{p}{(}\PY{n}{answer3}\PY{p}{)}\PY{p}{)}
         
         \PY{c+c1}{\PYZsh{}4}
         \PY{c+c1}{\PYZsh{}can\PYZsq{}t be computed as A\PYZsq{}s dimension is 3*3 whereas b transpose is 1*3, so they can\PYZsq{}t be multiplied together}
         
         \PY{c+c1}{\PYZsh{}5}
         \PY{c+c1}{\PYZsh{}can\PYZsq{}t be computed as b\PYZsq{}s dimension is 3*1 whereas A\PYZsq{}s dimension is 3*3, so they can\PYZsq{}t be multiplied together}
         
         \PY{c+c1}{\PYZsh{}6}
         \PY{n}{answer6}\PY{o}{=}\PY{n}{b}\PY{o}{.}\PY{n}{T}\PY{n+nd}{@A}
         \PY{n+nb}{print}\PY{p}{(}\PY{l+s+s1}{\PYZsq{}}\PY{l+s+s1}{answer 6 }\PY{l+s+se}{\PYZbs{}n}\PY{l+s+s1}{ }\PY{l+s+si}{\PYZob{}\PYZcb{}}\PY{l+s+s1}{\PYZsq{}}\PY{o}{.}\PY{n}{format}\PY{p}{(}\PY{n}{answer6}\PY{p}{)}\PY{p}{)}
         
         \PY{c+c1}{\PYZsh{}7}
         \PY{c+c1}{\PYZsh{}can\PYZsq{}t be computed as b\PYZsq{}s dimension is 3*1, they can\PYZsq{}t be multiplied together}
         
         \PY{c+c1}{\PYZsh{}8}
         \PY{n}{answer8}\PY{o}{=}\PY{n}{b}\PY{o}{.}\PY{n}{T}\PY{n+nd}{@b}
         \PY{n+nb}{print}\PY{p}{(}\PY{l+s+s1}{\PYZsq{}}\PY{l+s+s1}{answer 8 }\PY{l+s+se}{\PYZbs{}n}\PY{l+s+s1}{ }\PY{l+s+si}{\PYZob{}\PYZcb{}}\PY{l+s+s1}{\PYZsq{}}\PY{o}{.}\PY{n}{format}\PY{p}{(}\PY{n}{answer8}\PY{p}{)}\PY{p}{)}
         
         \PY{c+c1}{\PYZsh{}9}
         \PY{n}{answer9}\PY{o}{=}\PY{n}{b}\PY{n+nd}{@b}\PY{o}{.}\PY{n}{T}
         \PY{n+nb}{print}\PY{p}{(}\PY{l+s+s1}{\PYZsq{}}\PY{l+s+s1}{answer 9 }\PY{l+s+se}{\PYZbs{}n}\PY{l+s+s1}{ }\PY{l+s+si}{\PYZob{}\PYZcb{}}\PY{l+s+s1}{\PYZsq{}}\PY{o}{.}\PY{n}{format}\PY{p}{(}\PY{n}{answer9}\PY{p}{)}\PY{p}{)}
         
         \PY{c+c1}{\PYZsh{}10}
         \PY{n}{answer10}\PY{o}{=}\PY{n}{b}\PY{o}{+}\PY{n}{c}\PY{o}{.}\PY{n}{T}
         \PY{n+nb}{print}\PY{p}{(}\PY{l+s+s1}{\PYZsq{}}\PY{l+s+s1}{answer 10 }\PY{l+s+se}{\PYZbs{}n}\PY{l+s+s1}{ }\PY{l+s+si}{\PYZob{}\PYZcb{}}\PY{l+s+s1}{\PYZsq{}}\PY{o}{.}\PY{n}{format}\PY{p}{(}\PY{n}{answer10}\PY{p}{)}\PY{p}{)}
         
         \PY{c+c1}{\PYZsh{}11}
         \PY{c+c1}{\PYZsh{}can\PYZsq{}t be computed as b tranpose\PYZsq{}s dimension is 1*3, they can be multiplied together}
         
         \PY{c+c1}{\PYZsh{}12}
         \PY{n}{answer12}\PY{o}{=}\PY{n}{np}\PY{o}{.}\PY{n}{linalg}\PY{o}{.}\PY{n}{inv}\PY{p}{(}\PY{n}{A}\PY{p}{)}\PY{n+nd}{@b}
         \PY{n+nb}{print}\PY{p}{(}\PY{l+s+s1}{\PYZsq{}}\PY{l+s+s1}{answer 12 }\PY{l+s+se}{\PYZbs{}n}\PY{l+s+s1}{ }\PY{l+s+si}{\PYZob{}\PYZcb{}}\PY{l+s+s1}{\PYZsq{}}\PY{o}{.}\PY{n}{format}\PY{p}{(}\PY{n}{answer12}\PY{p}{)}\PY{p}{)}
         
         \PY{c+c1}{\PYZsh{}13}
         \PY{n}{answer13}\PY{o}{=}\PY{n}{A}\PY{o}{*}\PY{n}{A}
         \PY{n+nb}{print}\PY{p}{(}\PY{l+s+s1}{\PYZsq{}}\PY{l+s+s1}{answer 13 }\PY{l+s+se}{\PYZbs{}n}\PY{l+s+s1}{ }\PY{l+s+si}{\PYZob{}\PYZcb{}}\PY{l+s+s1}{\PYZsq{}}\PY{o}{.}\PY{n}{format}\PY{p}{(}\PY{n}{answer13}\PY{p}{)}\PY{p}{)}
         
         \PY{c+c1}{\PYZsh{}14}
         \PY{n}{answer14}\PY{o}{=}\PY{n}{b}\PY{o}{*}\PY{n}{c}
         \PY{n+nb}{print}\PY{p}{(}\PY{l+s+s1}{\PYZsq{}}\PY{l+s+s1}{answer 14 }\PY{l+s+se}{\PYZbs{}n}\PY{l+s+s1}{ }\PY{l+s+si}{\PYZob{}\PYZcb{}}\PY{l+s+s1}{\PYZsq{}}\PY{o}{.}\PY{n}{format}\PY{p}{(}\PY{n}{answer14}\PY{p}{)}\PY{p}{)}
\end{Verbatim}


    \begin{Verbatim}[commandchars=\\\{\}]
(3, 3)
(3, 1)
(3, 1)
answer 1 
 [[14 25 31]
 [25 45 56]
 [31 56 70]]
answer 2 
 [[14 25 31]
 [25 45 56]
 [31 56 70]]
answer 3 
 [[29]
 [50]
 [60]]
answer 6 
 [[29 50 60]]
answer 8 
 [[74]]
answer 9 
 [[ 1 -3 -8]
 [-3  9 24]
 [-8 24 64]]
answer 10 
 [[ 3 -4  5]
 [ 7  0  9]
 [12  5 14]]
answer 12 
 [[ 6.]
 [ 4.]
 [-5.]]
answer 13 
 [[ 1  4  9]
 [ 4 16 25]
 [ 9 25 36]]
answer 14 
 [[-4]
 [-9]
 [48]]

    \end{Verbatim}

    \subsection{6}\label{section}

\textbf{Eigenvectors and eigenvalues}. Eigenvectors and eigenvalues are
useful for some machine learning algorithms, but the concepts take time
to solidly grasp. For an intuitive review of these concepts, explore
this
\href{http://setosa.io/ev/eigenvectors-and-eigenvalues/}{interactive
website at Setosa.io}. Also, the series of linear algebra videos by
Grant Sanderson of 3Brown1Blue are excellent and can be viewed on
youtube
\href{https://www.youtube.com/playlist?list=PLZHQObOWTQDPD3MizzM2xVFitgF8hE_ab}{here}.

\begin{enumerate}
\def\labelenumi{\arabic{enumi}.}
\tightlist
\item
  Calculate the eigenvalues and corresponding eigenvectors of matrix
  \(\mathbf{A}\) above, from the last question.
\item
  Choose one of the eigenvector/eigenvalue pairs, \(\mathbf{v}\) and
  \(\lambda\), and show that
  \(\mathbf{A} \mathbf{v} = \lambda \mathbf{v}\). Also show that this
  relationship extends to higher orders:
  \(\mathbf{A} \mathbf{A} \mathbf{v} = \lambda^2 \mathbf{v}\)
\item
  Show that the eigenvectors are orthogonal to one another (e.g. their
  inner product is zero). This is true for real, symmetric matrices.
\end{enumerate}

    \textbf{ANSWER}

    \begin{Verbatim}[commandchars=\\\{\}]
{\color{incolor}In [{\color{incolor}103}]:} \PY{c+c1}{\PYZsh{}1}
          \PY{n}{eigenval}\PY{o}{=}\PY{n}{np}\PY{o}{.}\PY{n}{linalg}\PY{o}{.}\PY{n}{eigvals}\PY{p}{(}\PY{n}{A}\PY{p}{)}
          \PY{n+nb}{print}\PY{p}{(}\PY{l+s+s1}{\PYZsq{}}\PY{l+s+s1}{eigenvalues are }\PY{l+s+si}{\PYZob{}\PYZcb{}}\PY{l+s+s1}{\PYZsq{}}\PY{o}{.}\PY{n}{format}\PY{p}{(}\PY{n}{eigenval}\PY{p}{)}\PY{p}{)}
          \PY{n}{eigenvec}\PY{o}{=}\PY{n}{np}\PY{o}{.}\PY{n}{linalg}\PY{o}{.}\PY{n}{eig}\PY{p}{(}\PY{n}{A}\PY{p}{)}\PY{p}{[}\PY{l+m+mi}{1}\PY{p}{]}
          \PY{n+nb}{print}\PY{p}{(}\PY{l+s+s1}{\PYZsq{}}\PY{l+s+s1}{eigenvector is }\PY{l+s+si}{\PYZob{}\PYZcb{}}\PY{l+s+s1}{\PYZsq{}}\PY{o}{.}\PY{n}{format}\PY{p}{(}\PY{n}{eigenvec}\PY{p}{)}\PY{p}{)}
          
          \PY{c+c1}{\PYZsh{}2}
          \PY{n}{eigenvalnew}\PY{o}{=}\PY{n}{eigenval}\PY{p}{[}\PY{l+m+mi}{0}\PY{p}{]}
          \PY{n}{eigenvecnew}\PY{o}{=}\PY{n}{eigenvec}\PY{p}{[}\PY{p}{:}\PY{p}{,}\PY{l+m+mi}{0}\PY{p}{]}
          \PY{n}{Av}\PY{o}{=}\PY{n}{A}\PY{n+nd}{@eigenvecnew}
          \PY{n+nb}{print}\PY{p}{(}\PY{l+s+s1}{\PYZsq{}}\PY{l+s+s1}{Av is }\PY{l+s+si}{\PYZob{}\PYZcb{}}\PY{l+s+s1}{\PYZsq{}}\PY{o}{.}\PY{n}{format}\PY{p}{(}\PY{n}{Av}\PY{p}{)}\PY{p}{)}
          \PY{n}{lambdav}\PY{o}{=}\PY{n}{eigenvalnew}\PY{o}{*}\PY{n}{eigenvecnew}
          \PY{n+nb}{print}\PY{p}{(}\PY{l+s+s1}{\PYZsq{}}\PY{l+s+s1}{lambda*v is }\PY{l+s+si}{\PYZob{}\PYZcb{}}\PY{l+s+s1}{\PYZsq{}}\PY{o}{.}\PY{n}{format}\PY{p}{(}\PY{n}{lambdav}\PY{p}{)}\PY{p}{)}
          \PY{c+c1}{\PYZsh{}as we observe, Av=lambda*v}
          
          \PY{n}{AAv}\PY{o}{=}\PY{n}{A}\PY{n+nd}{@A}\PY{n+nd}{@eigenvecnew}
          \PY{n}{lambda2v}\PY{o}{=}\PY{n}{eigenvalnew}\PY{o}{*}\PY{n}{eigenvalnew}\PY{o}{*}\PY{n}{eigenvecnew}
          \PY{n+nb}{print}\PY{p}{(}\PY{l+s+s1}{\PYZsq{}}\PY{l+s+s1}{AAV is }\PY{l+s+si}{\PYZob{}\PYZcb{}}\PY{l+s+s1}{\PYZsq{}}\PY{o}{.}\PY{n}{format}\PY{p}{(}\PY{n}{AAv}\PY{p}{)}\PY{p}{)}
          \PY{n+nb}{print}\PY{p}{(}\PY{l+s+s1}{\PYZsq{}}\PY{l+s+s1}{lambda square v is }\PY{l+s+si}{\PYZob{}\PYZcb{}}\PY{l+s+s1}{\PYZsq{}}\PY{o}{.}\PY{n}{format}\PY{p}{(}\PY{n}{lambda2v}\PY{p}{)}\PY{p}{)}
          \PY{c+c1}{\PYZsh{}as we observe, AAv=lambda\PYZca{}2*v}
          
          \PY{c+c1}{\PYZsh{}3}
          \PY{n}{innerproduct1}\PY{o}{=}\PY{n}{np}\PY{o}{.}\PY{n}{inner}\PY{p}{(}\PY{n}{eigenvec}\PY{p}{[}\PY{p}{:}\PY{p}{,}\PY{l+m+mi}{0}\PY{p}{]}\PY{p}{,}\PY{n}{eigenvec}\PY{p}{[}\PY{p}{:}\PY{p}{,}\PY{l+m+mi}{1}\PY{p}{]}\PY{p}{)}
          \PY{n}{innerproduct2}\PY{o}{=}\PY{n}{np}\PY{o}{.}\PY{n}{inner}\PY{p}{(}\PY{n}{eigenvec}\PY{p}{[}\PY{p}{:}\PY{p}{,}\PY{l+m+mi}{1}\PY{p}{]}\PY{p}{,}\PY{n}{eigenvec}\PY{p}{[}\PY{p}{:}\PY{p}{,}\PY{l+m+mi}{2}\PY{p}{]}\PY{p}{)}
          \PY{n}{innerproduct3}\PY{o}{=}\PY{n}{np}\PY{o}{.}\PY{n}{inner}\PY{p}{(}\PY{n}{eigenvec}\PY{p}{[}\PY{p}{:}\PY{p}{,}\PY{l+m+mi}{0}\PY{p}{]}\PY{p}{,}\PY{n}{eigenvec}\PY{p}{[}\PY{p}{:}\PY{p}{,}\PY{l+m+mi}{2}\PY{p}{]}\PY{p}{)}
          \PY{n+nb}{print}\PY{p}{(}\PY{l+s+s1}{\PYZsq{}}\PY{l+s+s1}{innerproducts are }\PY{l+s+si}{\PYZob{}\PYZcb{}}\PY{l+s+s1}{ }\PY{l+s+si}{\PYZob{}\PYZcb{}}\PY{l+s+s1}{ }\PY{l+s+si}{\PYZob{}\PYZcb{}}\PY{l+s+s1}{\PYZsq{}}\PY{o}{.}\PY{n}{format}\PY{p}{(}\PY{n+nb}{round}\PY{p}{(}\PY{n}{innerproduct1}\PY{p}{,}\PY{l+m+mi}{1}\PY{p}{)}\PY{p}{,}\PY{n+nb}{round}\PY{p}{(}\PY{n}{innerproduct2}\PY{p}{,}\PY{l+m+mi}{1}\PY{p}{)}\PY{p}{,}\PY{n+nb}{round}\PY{p}{(}\PY{n}{innerproduct3}\PY{p}{,}\PY{l+m+mi}{1}\PY{p}{)}\PY{p}{)}\PY{p}{)}
          \PY{c+c1}{\PYZsh{}innerproducts are all 0, thus eigenvectors are orthogonal to each other}
\end{Verbatim}


    \begin{Verbatim}[commandchars=\\\{\}]
eigenvalues are [11.34481428 -0.51572947  0.17091519]
eigenvector is [[-0.32798528 -0.73697623  0.59100905]
 [-0.59100905 -0.32798528 -0.73697623]
 [-0.73697623  0.59100905  0.32798528]]
Av is [-3.72093206 -6.70488789 -8.36085845]
lambda*v is [-3.72093206 -6.70488789 -8.36085845]
AAV is [-42.2132832  -76.06570795 -94.85238636]
lambda square v is [-42.2132832  -76.06570795 -94.85238636]
innerproducts are -0.0 -0.0 -0.0

    \end{Verbatim}

    \section{Numerical Programming}\label{numerical-programming}

    \subsection{7}\label{section}

Speed comparison between vectorized and non-vectorized code. Begin by
creating an array of 10 million random numbers using the numpy
random.randn module. Compute the sum of the squares first in a for loop,
then using Numpy's \texttt{dot} module. Time how long it takes to
compute each and report the results and report the output. How many
times faster is the vectorized code than the for loop approach?

*Note: all code should be well commented, properly formatted, and your
answers should be output using the \texttt{print()} function as follows
(where the \# represents your answers, to a reasonable precision):

\texttt{Time\ {[}sec{]}\ (non-vectorized):\ \#\#\#\#\#\#}

\texttt{Time\ {[}sec{]}\ (vectorized):\ \ \ \ \ \#\#\#\#\#\#}

\texttt{The\ vectorized\ code\ is\ \#\#\#\#\#\ times\ faster\ than\ the\ vectorized\ code}

    \textbf{ANSWER}

    \begin{Verbatim}[commandchars=\\\{\}]
{\color{incolor}In [{\color{incolor}188}]:} \PY{k+kn}{import} \PY{n+nn}{numpy} \PY{k}{as} \PY{n+nn}{np}
          \PY{k+kn}{import} \PY{n+nn}{time}
          
          \PY{c+c1}{\PYZsh{} Generate the random samples}
          \PY{n}{sample}\PY{o}{=}\PY{n}{np}\PY{o}{.}\PY{n}{random}\PY{o}{.}\PY{n}{randn}\PY{p}{(}\PY{l+m+mi}{10000000}\PY{p}{)}
          \PY{c+c1}{\PYZsh{} Compute the sum of squares the non\PYZhy{}vectorized way (using a for loop)}
          \PY{n}{start1}\PY{o}{=}\PY{n}{time}\PY{o}{.}\PY{n}{clock}\PY{p}{(}\PY{p}{)}
          \PY{n}{result1}\PY{o}{=}\PY{l+m+mi}{0}
          \PY{k}{for} \PY{n}{each} \PY{o+ow}{in} \PY{n}{sample}\PY{p}{:}
              \PY{n}{result1}\PY{o}{=}\PY{n}{result1}\PY{o}{+}\PY{n}{each}\PY{o}{*}\PY{n}{each}
          \PY{n}{end1}\PY{o}{=}\PY{n}{time}\PY{o}{.}\PY{n}{clock}\PY{p}{(}\PY{p}{)}
          \PY{n}{elapsed1}\PY{o}{=}\PY{n}{end1}\PY{o}{\PYZhy{}}\PY{n}{start1}
          \PY{c+c1}{\PYZsh{} Compute the sum of squares the vectorized way (using numpy)}
          \PY{n}{start2}\PY{o}{=}\PY{n}{time}\PY{o}{.}\PY{n}{clock}\PY{p}{(}\PY{p}{)}
          \PY{n}{result2}\PY{o}{=}\PY{n}{np}\PY{o}{.}\PY{n}{sum}\PY{p}{(}\PY{n}{np}\PY{o}{.}\PY{n}{square}\PY{p}{(}\PY{n}{sample}\PY{p}{)}\PY{p}{)}
          \PY{n}{end2}\PY{o}{=}\PY{n}{time}\PY{o}{.}\PY{n}{clock}\PY{p}{(}\PY{p}{)}
          \PY{n}{elapsed2}\PY{o}{=}\PY{n}{end2}\PY{o}{\PYZhy{}}\PY{n}{start2}
          \PY{c+c1}{\PYZsh{} Print the results}
          \PY{n+nb}{print}\PY{p}{(}\PY{l+s+s1}{\PYZsq{}}\PY{l+s+s1}{Time [sec] (non\PYZhy{}vectorized):}\PY{l+s+si}{\PYZob{}\PYZcb{}}\PY{l+s+s1}{\PYZsq{}}\PY{o}{.}\PY{n}{format}\PY{p}{(}\PY{n}{elapsed1}\PY{p}{)}\PY{p}{)}
          \PY{n+nb}{print}\PY{p}{(}\PY{l+s+s1}{\PYZsq{}}\PY{l+s+s1}{Time [sec] (vectorized):}\PY{l+s+si}{\PYZob{}\PYZcb{}}\PY{l+s+s1}{\PYZsq{}}\PY{o}{.}\PY{n}{format}\PY{p}{(}\PY{n}{elapsed2}\PY{p}{)}\PY{p}{)}
          \PY{n}{diff}\PY{o}{=}\PY{n}{elapsed1}\PY{o}{/}\PY{n}{elapsed2}
          \PY{n+nb}{print}\PY{p}{(}\PY{l+s+s1}{\PYZsq{}}\PY{l+s+s1}{The vectorized code is }\PY{l+s+si}{\PYZob{}\PYZcb{}}\PY{l+s+s1}{ times faster than the non\PYZhy{}vectorized code}\PY{l+s+s1}{\PYZsq{}}\PY{o}{.}\PY{n}{format}\PY{p}{(}\PY{n}{diff}\PY{p}{)}\PY{p}{)}
\end{Verbatim}


    \begin{Verbatim}[commandchars=\\\{\}]
Time [sec] (non-vectorized):2.1382639999999924
Time [sec] (vectorized):0.16973000000000127
The vectorized code is 12.598032168738445 times faster than the non-vectorized code

    \end{Verbatim}

    \subsection{8}\label{section}

One popular Agile development framework is Scrum (a paradigm recommended
for data science projects). It emphasizes the continual evolution of
code for projects, becoming progressively better, but starting with a
quickly developed minimum viable product. This often means that code
written early on is not optimized, and that's a good thing - it's best
to get it to work first before optimizing. Imagine that you wrote the
following code during a sprint towards getting an end-to-end system
working. Vectorize the following code and show the difference in speed
between the current implementation and a vectorized version.

The function below computes the function \(f(x,y) = x^2 - 2 y^2\) and
determines whether this quantity is above or below a given threshold,
\texttt{thresh=0}. This is done for \(x,y \in \{-4,4\}\), over a
2,000-by-2,000 grid covering that domain.

\begin{enumerate}
\def\labelenumi{(\alph{enumi})}
\tightlist
\item
  Vectorize this code and demonstrate (as in the last exercise) the
  speed increase through vectorization and (b) plot the resulting data -
  both the function \(f(x,y)\) and the thresholded output - using
  \href{https://matplotlib.org/api/_as_gen/matplotlib.pyplot.imshow.html?highlight=matplotlib\%20pyplot\%20imshow\#matplotlib.pyplot.imshow}{\texttt{imshow}}
  from \texttt{matplotlib}.
\end{enumerate}

\emph{Hint: look at the \texttt{numpy}
\href{https://docs.scipy.org/doc/numpy-1.13.0/reference/generated/numpy.meshgrid.html}{\texttt{meshgrid}}
documentation}

    \begin{Verbatim}[commandchars=\\\{\}]
{\color{incolor}In [{\color{incolor}185}]:} \PY{k+kn}{import} \PY{n+nn}{numpy} \PY{k}{as} \PY{n+nn}{np}
          \PY{k+kn}{import} \PY{n+nn}{time}
          \PY{k+kn}{import} \PY{n+nn}{matplotlib}\PY{n+nn}{.}\PY{n+nn}{pyplot} \PY{k}{as} \PY{n+nn}{plt}
          
          \PY{c+c1}{\PYZsh{} Initialize variables for this exerise}
          \PY{n}{x}\PY{o}{=}\PY{n}{np}\PY{o}{.}\PY{n}{linspace}\PY{p}{(}\PY{o}{\PYZhy{}}\PY{l+m+mi}{4}\PY{p}{,}\PY{l+m+mi}{4}\PY{p}{,}\PY{l+m+mi}{2000}\PY{p}{)}
          \PY{n}{y}\PY{o}{=}\PY{n}{np}\PY{o}{.}\PY{n}{linspace}\PY{p}{(}\PY{o}{\PYZhy{}}\PY{l+m+mi}{4}\PY{p}{,}\PY{l+m+mi}{4}\PY{p}{,}\PY{l+m+mi}{2000}\PY{p}{)}
          
          \PY{c+c1}{\PYZsh{} Nonvectorized implementation}
          \PY{n}{start1}\PY{o}{=}\PY{n}{time}\PY{o}{.}\PY{n}{clock}\PY{p}{(}\PY{p}{)}
          \PY{n}{results}\PY{o}{=}\PY{n}{np}\PY{o}{.}\PY{n}{zeros}\PY{p}{(}\PY{p}{(}\PY{l+m+mi}{2000}\PY{p}{,}\PY{l+m+mi}{2000}\PY{p}{)}\PY{p}{)}
          \PY{n}{products}\PY{o}{=}\PY{n}{np}\PY{o}{.}\PY{n}{zeros}\PY{p}{(}\PY{p}{(}\PY{l+m+mi}{2000}\PY{p}{,}\PY{l+m+mi}{2000}\PY{p}{)}\PY{p}{)}
          \PY{k}{for} \PY{n}{i} \PY{o+ow}{in} \PY{n+nb}{range}\PY{p}{(}\PY{n+nb}{len}\PY{p}{(}\PY{n}{x}\PY{p}{)}\PY{p}{)}\PY{p}{:}
              \PY{k}{for} \PY{n}{j} \PY{o+ow}{in} \PY{n+nb}{range}\PY{p}{(}\PY{n+nb}{len}\PY{p}{(}\PY{n}{y}\PY{p}{)}\PY{p}{)}\PY{p}{:}
                  \PY{n}{product}\PY{o}{=}\PY{n}{x}\PY{p}{[}\PY{n}{i}\PY{p}{]}\PY{o}{*}\PY{n}{x}\PY{p}{[}\PY{n}{i}\PY{p}{]}\PY{o}{\PYZhy{}}\PY{l+m+mi}{2}\PY{o}{*}\PY{n}{y}\PY{p}{[}\PY{n}{j}\PY{p}{]}\PY{o}{*}\PY{n}{y}\PY{p}{[}\PY{n}{j}\PY{p}{]}
                  \PY{k}{if} \PY{n}{product}\PY{o}{\PYZgt{}}\PY{l+m+mi}{0}\PY{p}{:}
                      \PY{n}{results}\PY{p}{[}\PY{n}{j}\PY{p}{,}\PY{n}{i}\PY{p}{]}\PY{o}{=}\PY{k+kc}{True}
                      \PY{n}{products}\PY{p}{[}\PY{n}{j}\PY{p}{,}\PY{n}{i}\PY{p}{]}\PY{o}{=}\PY{n}{product}
                  \PY{k}{else}\PY{p}{:}
                      \PY{n}{results}\PY{p}{[}\PY{n}{j}\PY{p}{,}\PY{n}{i}\PY{p}{]}\PY{o}{=}\PY{k+kc}{False}
                      \PY{n}{products}\PY{p}{[}\PY{n}{j}\PY{p}{,}\PY{n}{i}\PY{p}{]}\PY{o}{=}\PY{n}{product}
          \PY{n}{end1}\PY{o}{=}\PY{n}{time}\PY{o}{.}\PY{n}{clock}\PY{p}{(}\PY{p}{)}
          \PY{n}{elapsed1}\PY{o}{=}\PY{n}{end1}\PY{o}{\PYZhy{}}\PY{n}{start1}
          
          \PY{c+c1}{\PYZsh{} Vectorized implementation}
          \PY{n}{start2}\PY{o}{=}\PY{n}{time}\PY{o}{.}\PY{n}{clock}\PY{p}{(}\PY{p}{)}
          \PY{n}{xx}\PY{p}{,}\PY{n}{yy}\PY{o}{=}\PY{n}{np}\PY{o}{.}\PY{n}{meshgrid}\PY{p}{(}\PY{n}{x}\PY{p}{,}\PY{n}{y}\PY{p}{)}
          \PY{n}{products2}\PY{o}{=}\PY{n}{np}\PY{o}{.}\PY{n}{square}\PY{p}{(}\PY{n}{xx}\PY{p}{)}\PY{o}{\PYZhy{}}\PY{l+m+mi}{2}\PY{o}{*}\PY{n}{np}\PY{o}{.}\PY{n}{square}\PY{p}{(}\PY{n}{yy}\PY{p}{)}
          \PY{n}{results2}\PY{o}{=}\PY{n}{products2}\PY{o}{\PYZgt{}}\PY{l+m+mi}{0}
          \PY{n}{end2}\PY{o}{=}\PY{n}{time}\PY{o}{.}\PY{n}{clock}\PY{p}{(}\PY{p}{)}
          \PY{n}{elapsed2}\PY{o}{=}\PY{n}{end2}\PY{o}{\PYZhy{}}\PY{n}{start2}
          
          \PY{c+c1}{\PYZsh{} Print the time for each and the speed increase}
          \PY{n+nb}{print}\PY{p}{(}\PY{l+s+s1}{\PYZsq{}}\PY{l+s+s1}{Time [sec] (non\PYZhy{}vectorized):}\PY{l+s+si}{\PYZob{}\PYZcb{}}\PY{l+s+s1}{\PYZsq{}}\PY{o}{.}\PY{n}{format}\PY{p}{(}\PY{n}{elapsed1}\PY{p}{)}\PY{p}{)}
          \PY{n+nb}{print}\PY{p}{(}\PY{l+s+s1}{\PYZsq{}}\PY{l+s+s1}{Time [sec] (vectorized):}\PY{l+s+si}{\PYZob{}\PYZcb{}}\PY{l+s+s1}{\PYZsq{}}\PY{o}{.}\PY{n}{format}\PY{p}{(}\PY{n}{elapsed2}\PY{p}{)}\PY{p}{)}
          \PY{n}{diff}\PY{o}{=}\PY{n}{elapsed1}\PY{o}{/}\PY{n}{elapsed2}
          \PY{n+nb}{print}\PY{p}{(}\PY{l+s+s1}{\PYZsq{}}\PY{l+s+s1}{The vectorized code is }\PY{l+s+si}{\PYZob{}\PYZcb{}}\PY{l+s+s1}{ times faster than the non\PYZhy{}vectorized code}\PY{l+s+s1}{\PYZsq{}}\PY{o}{.}\PY{n}{format}\PY{p}{(}\PY{n}{diff}\PY{p}{)}\PY{p}{)}
\end{Verbatim}


    \begin{Verbatim}[commandchars=\\\{\}]
Time [sec] (non-vectorized):5.2209900000000005
Time [sec] (vectorized):0.24316799999999716
The vectorized code is 21.47071160679062 times faster than the non-vectorized code

    \end{Verbatim}

    \begin{Verbatim}[commandchars=\\\{\}]
{\color{incolor}In [{\color{incolor}186}]:} \PY{c+c1}{\PYZsh{} Plot the result}
          \PY{n}{fig}\PY{p}{,} \PY{n}{axarr} \PY{o}{=} \PY{n}{plt}\PY{o}{.}\PY{n}{subplots}\PY{p}{(}\PY{l+m+mi}{2}\PY{p}{,} \PY{l+m+mi}{2}\PY{p}{)}
          \PY{n}{fig}\PY{o}{.}\PY{n}{suptitle}\PY{p}{(}\PY{l+s+s2}{\PYZdq{}}\PY{l+s+s2}{Plots of Results}\PY{l+s+s2}{\PYZdq{}}\PY{p}{,} \PY{n}{fontsize}\PY{o}{=}\PY{l+m+mi}{16}\PY{p}{)}
          
          \PY{n}{axarr}\PY{p}{[}\PY{l+m+mi}{0}\PY{p}{,} \PY{l+m+mi}{0}\PY{p}{]}\PY{o}{.}\PY{n}{imshow}\PY{p}{(}\PY{n}{products}\PY{p}{)}
          \PY{n}{axarr}\PY{p}{[}\PY{l+m+mi}{0}\PY{p}{,} \PY{l+m+mi}{0}\PY{p}{]}\PY{o}{.}\PY{n}{set\PYZus{}title}\PY{p}{(}\PY{l+s+s1}{\PYZsq{}}\PY{l+s+s1}{Function Result (non\PYZhy{}vec)}\PY{l+s+s1}{\PYZsq{}}\PY{p}{)}
          \PY{n}{axarr}\PY{p}{[}\PY{l+m+mi}{0}\PY{p}{,} \PY{l+m+mi}{1}\PY{p}{]}\PY{o}{.}\PY{n}{imshow}\PY{p}{(}\PY{n}{results}\PY{p}{)}
          \PY{n}{axarr}\PY{p}{[}\PY{l+m+mi}{0}\PY{p}{,} \PY{l+m+mi}{1}\PY{p}{]}\PY{o}{.}\PY{n}{set\PYZus{}title}\PY{p}{(}\PY{l+s+s1}{\PYZsq{}}\PY{l+s+s1}{Threshold Result (non\PYZhy{}vec)}\PY{l+s+s1}{\PYZsq{}}\PY{p}{)}
          \PY{n}{axarr}\PY{p}{[}\PY{l+m+mi}{1}\PY{p}{,} \PY{l+m+mi}{0}\PY{p}{]}\PY{o}{.}\PY{n}{imshow}\PY{p}{(}\PY{n}{products2}\PY{p}{)}
          \PY{n}{axarr}\PY{p}{[}\PY{l+m+mi}{1}\PY{p}{,} \PY{l+m+mi}{0}\PY{p}{]}\PY{o}{.}\PY{n}{set\PYZus{}title}\PY{p}{(}\PY{l+s+s1}{\PYZsq{}}\PY{l+s+s1}{Function Result (vec)}\PY{l+s+s1}{\PYZsq{}}\PY{p}{)}
          \PY{n}{axarr}\PY{p}{[}\PY{l+m+mi}{1}\PY{p}{,} \PY{l+m+mi}{1}\PY{p}{]}\PY{o}{.}\PY{n}{imshow}\PY{p}{(}\PY{n}{results2}\PY{p}{)}
          \PY{n}{axarr}\PY{p}{[}\PY{l+m+mi}{1}\PY{p}{,} \PY{l+m+mi}{1}\PY{p}{]}\PY{o}{.}\PY{n}{set\PYZus{}title}\PY{p}{(}\PY{l+s+s1}{\PYZsq{}}\PY{l+s+s1}{Threshold Result (vec)}\PY{l+s+s1}{\PYZsq{}}\PY{p}{)}
\end{Verbatim}


\begin{Verbatim}[commandchars=\\\{\}]
{\color{outcolor}Out[{\color{outcolor}186}]:} Text(0.5,1,'Threshold Result (vec)')
\end{Verbatim}
            
    \begin{center}
    \adjustimage{max size={0.9\linewidth}{0.9\paperheight}}{output_29_1.png}
    \end{center}
    { \hspace*{\fill} \\}
    
    \subsection{9}\label{section}

This exercise will walk through some basic numerical programming
exercises. 1. Synthesize \(n=10^4\) normally distributed data points
with mean \(\mu=2\) and a standard deviation of \(\sigma=1\). Call these
observations from a random variable \(X\), and call the vector of
observations that you generate, \(\textbf{x}\). 2. Calculate the mean
and standard deviation of \(\textbf{x}\) to validate (1) and provide the
result to a precision of four significant figures. 3. Plot a histogram
of the data in \(\textbf{x}\) with 30 bins 4. What is the 90th
percentile of \(\textbf{x}\)? The 90th percentile is the value below
which 90\% of observations can be found. 5. What is the 99th percentile
of \(\textbf{x}\)? 6. Now synthesize \(n=10^4\) normally distributed
data points with mean \(\mu=0\) and a standard deviation of
\(\sigma=3\). Call these observations from a random variable \(Y\), and
call the vector of observations that you generate, \(\textbf{y}\). 7.
Plot the histogram of the data in \(\textbf{y}\) on a (new) plot with
the histogram of \(\textbf{x}\), so that both histograms can be seen and
compared. 8. Using the observations from \(\textbf{x}\) and
\(\textbf{y}\), estimate \(E[XY]\)

    \textbf{ANSWER}

    \begin{Verbatim}[commandchars=\\\{\}]
{\color{incolor}In [{\color{incolor}206}]:} \PY{c+c1}{\PYZsh{} 1}
          \PY{n}{mu}\PY{p}{,}\PY{n}{sigma}\PY{o}{=}\PY{l+m+mi}{2}\PY{p}{,}\PY{l+m+mi}{1}
          \PY{n}{x}\PY{o}{=}\PY{n}{np}\PY{o}{.}\PY{n}{random}\PY{o}{.}\PY{n}{normal}\PY{p}{(}\PY{n}{mu}\PY{p}{,}\PY{n}{sigma}\PY{p}{,}\PY{l+m+mi}{10000}\PY{p}{)}
          \PY{n+nb}{print}\PY{p}{(}\PY{n}{x}\PY{p}{)}
          
          \PY{c+c1}{\PYZsh{} 2}
          \PY{n}{mean}\PY{o}{=}\PY{n}{x}\PY{o}{.}\PY{n}{mean}\PY{p}{(}\PY{p}{)}
          \PY{n}{sd}\PY{o}{=}\PY{n}{x}\PY{o}{.}\PY{n}{std}\PY{p}{(}\PY{p}{)}
          \PY{n+nb}{print}\PY{p}{(}\PY{l+s+s1}{\PYZsq{}}\PY{l+s+s1}{Mean of x is }\PY{l+s+si}{\PYZob{}:.4G\PYZcb{}}\PY{l+s+s1}{\PYZsq{}}\PY{o}{.}\PY{n}{format}\PY{p}{(}\PY{n}{mean}\PY{p}{)}\PY{p}{)}
          \PY{n+nb}{print}\PY{p}{(}\PY{l+s+s1}{\PYZsq{}}\PY{l+s+s1}{Standard Deviation of x is }\PY{l+s+si}{\PYZob{}:.4G\PYZcb{}}\PY{l+s+s1}{\PYZsq{}}\PY{o}{.}\PY{n}{format}\PY{p}{(}\PY{n}{sd}\PY{p}{)}\PY{p}{)}
          
          \PY{c+c1}{\PYZsh{} 3}
          \PY{n}{fig}\PY{o}{=}\PY{n}{plt}\PY{o}{.}\PY{n}{figure}\PY{p}{(}\PY{p}{)}
          \PY{n}{fig}\PY{o}{.}\PY{n}{suptitle}\PY{p}{(}\PY{l+s+s2}{\PYZdq{}}\PY{l+s+s2}{Histogram of x with 30 bins}\PY{l+s+s2}{\PYZdq{}}\PY{p}{,} \PY{n}{fontsize}\PY{o}{=}\PY{l+m+mi}{16}\PY{p}{)}
          \PY{n}{plt}\PY{o}{.}\PY{n}{hist}\PY{p}{(}\PY{n}{x}\PY{p}{,}\PY{n}{bins}\PY{o}{=}\PY{l+m+mi}{30}\PY{p}{)}
          \PY{n}{plt}\PY{o}{.}\PY{n}{xlabel}\PY{p}{(}\PY{l+s+s1}{\PYZsq{}}\PY{l+s+s1}{value}\PY{l+s+s1}{\PYZsq{}}\PY{p}{)}
          \PY{n}{plt}\PY{o}{.}\PY{n}{ylabel}\PY{p}{(}\PY{l+s+s1}{\PYZsq{}}\PY{l+s+s1}{frequency}\PY{l+s+s1}{\PYZsq{}}\PY{p}{)}
          
          \PY{c+c1}{\PYZsh{} 4}
          \PY{n}{percentile90}\PY{o}{=}\PY{n}{np}\PY{o}{.}\PY{n}{percentile}\PY{p}{(}\PY{n}{x}\PY{p}{,}\PY{l+m+mi}{90}\PY{p}{)}
          \PY{n+nb}{print}\PY{p}{(}\PY{l+s+s1}{\PYZsq{}}\PY{l+s+s1}{90 percentile in x is }\PY{l+s+si}{\PYZob{}:.2f\PYZcb{}}\PY{l+s+s1}{\PYZsq{}}\PY{o}{.}\PY{n}{format}\PY{p}{(}\PY{n}{percentile90}\PY{p}{)}\PY{p}{)}
          
          \PY{c+c1}{\PYZsh{} 5}
          \PY{n}{percentile99}\PY{o}{=}\PY{n}{np}\PY{o}{.}\PY{n}{percentile}\PY{p}{(}\PY{n}{x}\PY{p}{,}\PY{l+m+mi}{99}\PY{p}{)}
          \PY{n+nb}{print}\PY{p}{(}\PY{l+s+s1}{\PYZsq{}}\PY{l+s+s1}{99 percentile in x is }\PY{l+s+si}{\PYZob{}:.2f\PYZcb{}}\PY{l+s+s1}{\PYZsq{}}\PY{o}{.}\PY{n}{format}\PY{p}{(}\PY{n}{percentile99}\PY{p}{)}\PY{p}{)}
          
          
          \PY{c+c1}{\PYZsh{} 6}
          \PY{n}{muy}\PY{p}{,}\PY{n}{sigmay}\PY{o}{=}\PY{l+m+mi}{0}\PY{p}{,}\PY{l+m+mi}{3}
          \PY{n}{y}\PY{o}{=}\PY{n}{np}\PY{o}{.}\PY{n}{random}\PY{o}{.}\PY{n}{normal}\PY{p}{(}\PY{n}{muy}\PY{p}{,}\PY{n}{sigmay}\PY{p}{,}\PY{l+m+mi}{10000}\PY{p}{)}
          \PY{n+nb}{print}\PY{p}{(}\PY{n}{y}\PY{p}{)}
          
          \PY{c+c1}{\PYZsh{} 7}
          \PY{n}{fignew}\PY{o}{=}\PY{n}{plt}\PY{o}{.}\PY{n}{figure}\PY{p}{(}\PY{p}{)}
          \PY{n}{ax1}\PY{o}{=}\PY{n}{plt}\PY{o}{.}\PY{n}{hist}\PY{p}{(}\PY{p}{[}\PY{n}{x}\PY{p}{,}\PY{n}{y}\PY{p}{]}\PY{p}{,}\PY{n}{bins}\PY{o}{=}\PY{l+m+mi}{30}\PY{p}{,}\PY{n}{label}\PY{o}{=}\PY{p}{[}\PY{l+s+s1}{\PYZsq{}}\PY{l+s+s1}{x}\PY{l+s+s1}{\PYZsq{}}\PY{p}{,}\PY{l+s+s1}{\PYZsq{}}\PY{l+s+s1}{y}\PY{l+s+s1}{\PYZsq{}}\PY{p}{]}\PY{p}{)}
          \PY{n}{plt}\PY{o}{.}\PY{n}{legend}\PY{p}{(}\PY{n}{loc}\PY{o}{=}\PY{l+s+s1}{\PYZsq{}}\PY{l+s+s1}{upper left}\PY{l+s+s1}{\PYZsq{}}\PY{p}{)}
          \PY{n}{plt}\PY{o}{.}\PY{n}{title}\PY{p}{(}\PY{l+s+s1}{\PYZsq{}}\PY{l+s+s1}{Histogram of x and y}\PY{l+s+s1}{\PYZsq{}}\PY{p}{)}
          \PY{n}{plt}\PY{o}{.}\PY{n}{xlabel}\PY{p}{(}\PY{l+s+s1}{\PYZsq{}}\PY{l+s+s1}{value}\PY{l+s+s1}{\PYZsq{}}\PY{p}{)}
          \PY{n}{plt}\PY{o}{.}\PY{n}{ylabel}\PY{p}{(}\PY{l+s+s1}{\PYZsq{}}\PY{l+s+s1}{frequency}\PY{l+s+s1}{\PYZsq{}}\PY{p}{)}
          
          
          \PY{c+c1}{\PYZsh{} 8}
          \PY{n}{expectation}\PY{o}{=}\PY{n}{np}\PY{o}{.}\PY{n}{mean}\PY{p}{(}\PY{n}{x}\PY{p}{)}\PY{o}{*}\PY{n}{np}\PY{o}{.}\PY{n}{mean}\PY{p}{(}\PY{n}{y}\PY{p}{)}
          \PY{n+nb}{print}\PY{p}{(}\PY{l+s+s1}{\PYZsq{}}\PY{l+s+s1}{E[XY] is }\PY{l+s+si}{\PYZob{}:.2f\PYZcb{}}\PY{l+s+s1}{\PYZsq{}}\PY{o}{.}\PY{n}{format}\PY{p}{(}\PY{n}{expectation}\PY{p}{)}\PY{p}{)}
\end{Verbatim}


    \begin{Verbatim}[commandchars=\\\{\}]
[3.26708944 0.94274575 0.67640309 {\ldots} 2.08970604 0.61739245 2.18430829]
Mean of x is 2.001
Standard Deviation of x is 1.008
90 percentile in x is 3.29
99 percentile in x is 4.32
[ 2.75991201 -1.809132    2.60463482 {\ldots} -0.67717308 -2.32114893
  0.96454609]
E[XY] is -0.04

    \end{Verbatim}

    \begin{center}
    \adjustimage{max size={0.9\linewidth}{0.9\paperheight}}{output_32_1.png}
    \end{center}
    { \hspace*{\fill} \\}
    
    \begin{center}
    \adjustimage{max size={0.9\linewidth}{0.9\paperheight}}{output_32_2.png}
    \end{center}
    { \hspace*{\fill} \\}
    
    \subsection{10}\label{section}

Estimate the integral of the function \(f(x)\) on the interval
\(0\leq x < 2.5\) assuming we only know the following points from \(f\):

\emph{Table 1. Dataset containing n=5 observations}

\begin{longtable}[]{@{}llllll@{}}
\toprule
\(x_i\) & 0.0 & 0.5 & 1.0 & 1.5 & 2.0\tabularnewline
\midrule
\endhead
\(y_i\) & 6 & 7 & 8 & 4 & 1\tabularnewline
\bottomrule
\end{longtable}

    \textbf{ANSWER}

    \begin{Verbatim}[commandchars=\\\{\}]
{\color{incolor}In [{\color{incolor}207}]:} \PY{n}{x}\PY{o}{=}\PY{n}{np}\PY{o}{.}\PY{n}{array}\PY{p}{(}\PY{p}{[}\PY{l+m+mf}{0.0}\PY{p}{,}\PY{l+m+mf}{0.5}\PY{p}{,}\PY{l+m+mf}{1.0}\PY{p}{,}\PY{l+m+mf}{1.5}\PY{p}{,}\PY{l+m+mf}{2.0}\PY{p}{]}\PY{p}{)}
          \PY{n}{y}\PY{o}{=}\PY{n}{np}\PY{o}{.}\PY{n}{array}\PY{p}{(}\PY{p}{[}\PY{l+m+mi}{6}\PY{p}{,}\PY{l+m+mi}{7}\PY{p}{,}\PY{l+m+mi}{8}\PY{p}{,}\PY{l+m+mi}{4}\PY{p}{,}\PY{l+m+mi}{1}\PY{p}{]}\PY{p}{)}
          
          \PY{c+c1}{\PYZsh{}using left riemann sum}
          \PY{n}{area}\PY{o}{=}\PY{l+m+mi}{6}\PY{o}{*}\PY{l+m+mf}{0.5}\PY{o}{+}\PY{l+m+mi}{7}\PY{o}{*}\PY{l+m+mf}{0.5}\PY{o}{+}\PY{l+m+mi}{8}\PY{o}{*}\PY{l+m+mf}{0.5}\PY{o}{+}\PY{l+m+mi}{4}\PY{o}{*}\PY{l+m+mf}{0.5}
          \PY{n+nb}{print}\PY{p}{(}\PY{l+s+s1}{\PYZsq{}}\PY{l+s+s1}{Integral estimation is }\PY{l+s+si}{\PYZob{}:.2f\PYZcb{}}\PY{l+s+s1}{\PYZsq{}}\PY{o}{.}\PY{n}{format}\PY{p}{(}\PY{n}{area}\PY{p}{)}\PY{p}{)}
\end{Verbatim}


    \begin{Verbatim}[commandchars=\\\{\}]
Integral estimation is 12.50

    \end{Verbatim}

    \section{Version Control via Git}\label{version-control-via-git}

    \subsection{11}\label{section}

Complete the
\href{https://www.atlassian.com/git/tutorials/what-is-version-control}{Atlassian
Git tutorial}, specifically the following sections. Try each concept
that's presented. For this tutorial, instead of using BitBucket, use
Github. Create a github account here if you don't already have one:
https://github.com/ 1.
\href{https://www.atlassian.com/git/tutorials/what-is-version-control}{What
is version control} 2.
\href{https://www.atlassian.com/git/tutorials/what-is-git}{What is Git}
3. \href{https://www.atlassian.com/git/tutorials/install-git}{Install
Git} 4.
\href{https://www.atlassian.com/git/tutorials/install-git}{Setting up a
repository} 5.
\href{https://www.atlassian.com/git/tutorials/saving-changes}{Saving
changes} 6.
\href{https://www.atlassian.com/git/tutorials/inspecting-a-repository}{Inspecting
a repository} 7.
\href{https://www.atlassian.com/git/tutorials/undoing-changes}{Undoing
changes} 8.
\href{https://www.atlassian.com/git/tutorials/rewriting-history}{Rewriting
history} 9.
\href{https://www.atlassian.com/git/tutorials/syncing}{Syncing} 10.
\href{https://www.atlassian.com/git/tutorials/making-a-pull-request}{Making
a pull request} 11.
\href{https://www.atlassian.com/git/tutorials/using-branches}{Using
branches} 12.
\href{https://www.atlassian.com/git/tutorials/comparing-workflows}{Comparing
workflows}

For your answer, affirm that you either completed the tutorial or have
previous experience with all of the concepts above. Do this by typing
your name below and selecting the situation that applies from the two
options in brackets.

    \textbf{ANSWER}

\emph{I, {[}\textbf{Emma Sun}{]}, affirm that I have {[}\textbf{I have
previous experience that covers all the content in this tutorial}{]}}

    \subsection{12}\label{section}

Using Github to create a static HTML website: 1. Create a branch in your
\texttt{machine-learning-course} repo called "gh-pages" and checkout
that branch (this will provide an example of how to create a simple
static website using \href{https://pages.github.com/}{Github Pages}) 2.
Create a file called "index.html" with the contents "Hello World" and
add, commit, and push it to that branch. 3. Submit the following: (a) a
link to your github repository and (b) a link to your new "Hello World"
website. The latter should be at the address
https://{[}USERNAME{]}.github.io/ECE590-assignment0 (where
{[}USERNAME{]} is your github username).

    \textbf{ANSWER}

    a)github repo: https://github.com/xuecongsun b)html website:
https://xuecongsun.github.io/machine-learning-course

    \section{Exploratory Data Analysis}\label{exploratory-data-analysis}

\subsection{13}\label{section}

Here you'll bring together some of the individual skills that you
demonstrated above and create a Jupyter notebook based blog post on data
analysis.

\begin{enumerate}
\def\labelenumi{\arabic{enumi}.}
\tightlist
\item
  Find a dataset that interests you and relates to a question or problem
  that you find intriguing
\item
  Using a Jupyter notebook, describe the dataset, the source of the
  data, and the reason the dataset was of interest.
\item
  Check the data and see if they need to be cleaned: are there missing
  values? Are there clearly erroneous values? Do two tables need to be
  merged together? Clean the data so it can be visualized.
\item
  Plot the data, demonstrating interesting features that you discover.
  Are there any relationships between variables that were surprising or
  patterns that emerged? Please exercise creativity and curiosity in
  your plots.
\item
  What insights are you able to take away from exploring the data? Is
  there a reason why analyzing the dataset you chose is particularly
  interesting or important? Summarize this as if your target audience
  was the readership of a major news organization - boil down your
  findings in a way that is accessible, but still accurate.
\item
  Create a public repository on your github account titled
  "machine-learning-course". In it, create a readme file that contains
  the heading "ECE590: Introductory Machine Learning for Data Science".
  Add, commit, and push that Jupyter notebook to the master branch.
  Provide the link to the that post here.
\end{enumerate}

    \textbf{ANSWER}

    https://github.com/xuecongsun/machine-learning-course


    % Add a bibliography block to the postdoc
    
    
    
    \end{document}
